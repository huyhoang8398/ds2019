\documentclass[]{article}
\usepackage{lmodern}
\usepackage{amssymb,amsmath}
\usepackage{ifxetex,ifluatex}
\usepackage{fixltx2e} % provides \textsubscript
\ifnum 0\ifxetex 1\fi\ifluatex 1\fi=0 % if pdftex
  \usepackage[T1]{fontenc}
  \usepackage[utf8]{inputenc}
\else % if luatex or xelatex
  \ifxetex
    \usepackage{mathspec}
  \else
    \usepackage{fontspec}
  \fi
  \defaultfontfeatures{Ligatures=TeX,Scale=MatchLowercase}
\fi
% use upquote if available, for straight quotes in verbatim environments
\IfFileExists{upquote.sty}{\usepackage{upquote}}{}
% use microtype if available
\IfFileExists{microtype.sty}{%
\usepackage{microtype}
\UseMicrotypeSet[protrusion]{basicmath} % disable protrusion for tt fonts
}{}
\usepackage[unicode=true]{hyperref}
\hypersetup{
            pdftitle={TCP File transfer},
            pdfauthor={ DO Duy Huy Hoang PHAM Son Tung Tran Duc Manh Pham Minh Kien},
            pdfborder={0 0 0},
            breaklinks=true}
\urlstyle{same}  % don't use monospace font for urls
\usepackage{color}
\usepackage{fancyvrb}
\newcommand{\VerbBar}{|}
\newcommand{\VERB}{\Verb[commandchars=\\\{\}]}
\DefineVerbatimEnvironment{Highlighting}{Verbatim}{commandchars=\\\{\}}
% Add ',fontsize=\small' for more characters per line
\newenvironment{Shaded}{}{}
\newcommand{\KeywordTok}[1]{\textcolor[rgb]{0.00,0.44,0.13}{\textbf{{#1}}}}
\newcommand{\DataTypeTok}[1]{\textcolor[rgb]{0.56,0.13,0.00}{{#1}}}
\newcommand{\DecValTok}[1]{\textcolor[rgb]{0.25,0.63,0.44}{{#1}}}
\newcommand{\BaseNTok}[1]{\textcolor[rgb]{0.25,0.63,0.44}{{#1}}}
\newcommand{\FloatTok}[1]{\textcolor[rgb]{0.25,0.63,0.44}{{#1}}}
\newcommand{\ConstantTok}[1]{\textcolor[rgb]{0.53,0.00,0.00}{{#1}}}
\newcommand{\CharTok}[1]{\textcolor[rgb]{0.25,0.44,0.63}{{#1}}}
\newcommand{\SpecialCharTok}[1]{\textcolor[rgb]{0.25,0.44,0.63}{{#1}}}
\newcommand{\StringTok}[1]{\textcolor[rgb]{0.25,0.44,0.63}{{#1}}}
\newcommand{\VerbatimStringTok}[1]{\textcolor[rgb]{0.25,0.44,0.63}{{#1}}}
\newcommand{\SpecialStringTok}[1]{\textcolor[rgb]{0.73,0.40,0.53}{{#1}}}
\newcommand{\ImportTok}[1]{{#1}}
\newcommand{\CommentTok}[1]{\textcolor[rgb]{0.38,0.63,0.69}{\textit{{#1}}}}
\newcommand{\DocumentationTok}[1]{\textcolor[rgb]{0.73,0.13,0.13}{\textit{{#1}}}}
\newcommand{\AnnotationTok}[1]{\textcolor[rgb]{0.38,0.63,0.69}{\textbf{\textit{{#1}}}}}
\newcommand{\CommentVarTok}[1]{\textcolor[rgb]{0.38,0.63,0.69}{\textbf{\textit{{#1}}}}}
\newcommand{\OtherTok}[1]{\textcolor[rgb]{0.00,0.44,0.13}{{#1}}}
\newcommand{\FunctionTok}[1]{\textcolor[rgb]{0.02,0.16,0.49}{{#1}}}
\newcommand{\VariableTok}[1]{\textcolor[rgb]{0.10,0.09,0.49}{{#1}}}
\newcommand{\ControlFlowTok}[1]{\textcolor[rgb]{0.00,0.44,0.13}{\textbf{{#1}}}}
\newcommand{\OperatorTok}[1]{\textcolor[rgb]{0.40,0.40,0.40}{{#1}}}
\newcommand{\BuiltInTok}[1]{{#1}}
\newcommand{\ExtensionTok}[1]{{#1}}
\newcommand{\PreprocessorTok}[1]{\textcolor[rgb]{0.74,0.48,0.00}{{#1}}}
\newcommand{\AttributeTok}[1]{\textcolor[rgb]{0.49,0.56,0.16}{{#1}}}
\newcommand{\RegionMarkerTok}[1]{{#1}}
\newcommand{\InformationTok}[1]{\textcolor[rgb]{0.38,0.63,0.69}{\textbf{\textit{{#1}}}}}
\newcommand{\WarningTok}[1]{\textcolor[rgb]{0.38,0.63,0.69}{\textbf{\textit{{#1}}}}}
\newcommand{\AlertTok}[1]{\textcolor[rgb]{1.00,0.00,0.00}{\textbf{{#1}}}}
\newcommand{\ErrorTok}[1]{\textcolor[rgb]{1.00,0.00,0.00}{\textbf{{#1}}}}
\newcommand{\NormalTok}[1]{{#1}}
\usepackage{graphicx,grffile}
\makeatletter
\def\maxwidth{\ifdim\Gin@nat@width>\linewidth\linewidth\else\Gin@nat@width\fi}
\def\maxheight{\ifdim\Gin@nat@height>\textheight\textheight\else\Gin@nat@height\fi}
\makeatother
% Scale images if necessary, so that they will not overflow the page
% margins by default, and it is still possible to overwrite the defaults
% using explicit options in \includegraphics[width, height, ...]{}
\setkeys{Gin}{width=\maxwidth,height=\maxheight,keepaspectratio}
\IfFileExists{parskip.sty}{%
\usepackage{parskip}
}{% else
\setlength{\parindent}{0pt}
\setlength{\parskip}{6pt plus 2pt minus 1pt}
}
\setlength{\emergencystretch}{3em}  % prevent overfull lines
\providecommand{\tightlist}{%
  \setlength{\itemsep}{0pt}\setlength{\parskip}{0pt}}
\setcounter{secnumdepth}{0}
% Redefines (sub)paragraphs to behave more like sections
\ifx\paragraph\undefined\else
\let\oldparagraph\paragraph
\renewcommand{\paragraph}[1]{\oldparagraph{#1}\mbox{}}
\fi
\ifx\subparagraph\undefined\else
\let\oldsubparagraph\subparagraph
\renewcommand{\subparagraph}[1]{\oldsubparagraph{#1}\mbox{}}
\fi

\title{TCP File transfer}
\providecommand{\subtitle}[1]{}
\subtitle{Report}
\author{\textbf{Member} \newline DO Duy Huy Hoang \newline PHAM Son Tung
\newline Tran Duc Manh \newline Pham Minh Kien
\newline \newline \textit{University of Science and Technology Hanoi}
\newline \textit{ICT Department}}
\date{2019-03-4}

\begin{document}
\maketitle

\newpage{}

\tableofcontents
\newpage{}

\subsection{Designing Protocol}\label{designing-protocol}

\includegraphics[width=0.50000\textwidth]{protocol.png} * We used TCP
sockets to establish a connection between the server and the client
program.

\subsection{System Organizing}\label{system-organizing}

\includegraphics{overview.png}\{ width = 50\%\} * We create a socket
first and also a folder for saving the file before executing the file
transfer. The server then binds to a port and listens to the client. *
The server side will get the file name and save in tempFile. * The
client will get the length of file's content + filename

\subsection{Detail Figure}\label{detail-figure}

\includegraphics[width=0.50000\textwidth]{server.png}\\
\includegraphics[width=0.50000\textwidth]{client.png}

\subsection{Code implementation}\label{code-implementation}

\subsubsection{Server side}\label{server-side}

\begin{itemize}
\item
  Create a folder to save the file from the client

\begin{Shaded}
\begin{Highlighting}[]
  \NormalTok{system(}\StringTok{"rm -rf /home/huyhoang8398/ds2019/Lab1.TCP/filereceived"}\NormalTok{);}
  \NormalTok{system(}\StringTok{"mkdir filereceived"}\NormalTok{);}
  \DataTypeTok{char} \NormalTok{directoryTemp[] = }\StringTok{"/home/huyhoang8398/ds2019/Lab1.TCP/filereceived/"}\NormalTok{;}
\end{Highlighting}
\end{Shaded}
\item
  To save the content of file

\begin{Shaded}
\begin{Highlighting}[]
    \NormalTok{read(cli, s, }\KeywordTok{sizeof}\NormalTok{(s));}

    \CommentTok{//get the file name and save in tempFile}
    \ControlFlowTok{for} \NormalTok{(}\DataTypeTok{int} \NormalTok{i = }\DecValTok{0}\NormalTok{; i < strlen(s); ++i)}
    \NormalTok{\{}
      \ControlFlowTok{if}\NormalTok{(s[i]==newline[}\DecValTok{0}\NormalTok{])\{}
        \NormalTok{index = i;}
        \ControlFlowTok{break}\NormalTok{;}
      \NormalTok{\}}\ControlFlowTok{else} \NormalTok{tempFile[i]=s[i];}
    \NormalTok{\}}
    \NormalTok{printf(}\StringTok{"}\SpecialCharTok{%s\textbackslash{}n}\StringTok{"}\NormalTok{, tempFile);}
    \NormalTok{strcat(directoryTemp,tempFile);}
    \NormalTok{printf(}\StringTok{"Save to }\SpecialCharTok{%s\textbackslash{}n}\StringTok{"}\NormalTok{, directoryTemp);}
    \CommentTok{//get content of the file}
    \DataTypeTok{char} \NormalTok{*tempContent;}
    \NormalTok{tempContent = (}\DataTypeTok{char} \NormalTok{*) malloc(}\KeywordTok{sizeof}\NormalTok{(}\DataTypeTok{char}\NormalTok{)*strlen(s));}
    \DataTypeTok{int} \NormalTok{j=}\DecValTok{0}\NormalTok{;}
    \ControlFlowTok{for} \NormalTok{(}\DataTypeTok{int} \NormalTok{i = index}\DecValTok{+1}\NormalTok{; i < strlen(s); ++i)\{}
      \NormalTok{tempContent[j++]=s[i];                            }
    \NormalTok{\}}
    \NormalTok{printf(}\StringTok{"File content : }\SpecialCharTok{\textbackslash{}n%s\textbackslash{}n}\StringTok{"}\NormalTok{,tempContent);}
    \NormalTok{f=fopen(directoryTemp,}\StringTok{"w+"}\NormalTok{); }\CommentTok{// open file to read }

    \NormalTok{fprintf(f, }\StringTok{"}\SpecialCharTok{%s}\StringTok{"}\NormalTok{,tempContent);}
    \NormalTok{fflush(stdin);}
    \NormalTok{fclose(f); }\CommentTok{// close to save file to the directory filereceive}
\end{Highlighting}
\end{Shaded}
\end{itemize}

\subsubsection{Client side}\label{client-side}

\begin{itemize}
\tightlist
\item
  Sending a file to the server.
\end{itemize}

\begin{Shaded}
\begin{Highlighting}[]
  \CommentTok{//open file to read using filename}
  \NormalTok{f = fopen(filename,}\StringTok{"r"}\NormalTok{);}

  \CommentTok{//get the length of file's content + filename}
  \DataTypeTok{size_t} \NormalTok{pos = ftell(f);    }\CommentTok{// Current position}
  \NormalTok{fseek(f, }\DecValTok{0}\NormalTok{, SEEK_END);    }\CommentTok{// Go to end}
  \DataTypeTok{size_t} \NormalTok{length = ftell(f); }\CommentTok{// read the position which is the size}
  \NormalTok{fseek(f, pos, SEEK_SET);}
  \DataTypeTok{int} \NormalTok{lengthint = length+strlen(filename);}

  \CommentTok{//allocate data into string}
  \NormalTok{s = (}\DataTypeTok{char} \NormalTok{*) malloc(lengthint*}\KeywordTok{sizeof}\NormalTok{(}\DataTypeTok{char}\NormalTok{));}
  \CommentTok{//copy filename to the first part of s}
  \NormalTok{strcpy(s,filename);}
  \CommentTok{//append a new line in s}
  \NormalTok{strcat(s,}\StringTok{"}\SpecialCharTok{\textbackslash{}n}\StringTok{"}\NormalTok{);}
  \CommentTok{//index for s, with value of filename}
  \NormalTok{temp = strlen(filename)}\DecValTok{+1}\NormalTok{;}
  \CommentTok{//save content of file into s}
  \ControlFlowTok{if} \NormalTok{(f)\{}
    \ControlFlowTok{while}\NormalTok{((c = getc(f))!=EOF)}
      \NormalTok{s[temp++] = c;}
  \NormalTok{\}}

  \NormalTok{fclose(f);}
  \FunctionTok{connect}\NormalTok{(serv, (}\KeywordTok{struct} \NormalTok{sockaddr *)&ad, ad_length);}
  \NormalTok{write(serv, s, strlen(s) + }\DecValTok{1}\NormalTok{);}
\end{Highlighting}
\end{Shaded}

\end{document}
